%!TEX root=../seke.tex
% mainfile: ../seke.tex

\vspace*{-.15in}
\section{Introduction}
\vspace*{-.05in}

Performance problems such as high response times in software applications have a significant effect on the customer’s satisfaction. The explosive growth of the Internet has contributed to the increased need for applications that perform at an appropriate speed. Performance problems are often detected late in the application life cycle, and the later they are discovered, the greater the cost to fix them. The use of stress testing is an increasingly common practice owing to the increasing number of users. In this scenario, the inadequate treatment of a workload generated by concurrent or simultaneous access due to several users can result in highly critical failures and negatively affect the customers perception of the company \cite{Draheim2006b} \cite{Jiang2010} \cite{Molyneaux2009} \cite{Wert2014}. 

The use of stress testing is an increasingly common practice owing to the increasing number of users. In this scenario, the inadequate treatment of a workload generated by concurrent or simultaneous access due to several users can result in highly critical failures and negatively affect the customers perception of the company \cite{Draheim2006b} \cite{Jiang2010}. 

Stress software testing is a expensive and difficult activity. The exponential
growth in the complexity of software makes the cost of testing has only continued to rise. Test case generation can be seen as a search problem. The test adequacy criterion is transformed into a fitness function and a set of solutions in the search
space are evaluated with respect to the fitness function using a metaheuristic search technique. Search-based software testing is the application of metaheuristic search techniques to generate software
tests cases or perform test execution \cite{Afzal2009a} \cite{Gay}.


Stress Search-based testing is seen as a promising approach to verifying timing constraints \cite{Afzal2009a}. A common objective of a stress search-based test is to find  scenarios that produce execution times that violate the specified timing constraints \cite{Sullivan}. 

Experiments involving stress search based tests are inherently complex and typically time-consuming to set up and
execute. Such experiments are also extremely difficult to
repeat. People who might want to duplicate published results, for example, must devote substantial resources to setting up and the environmental conditions are likely to be substantially different. Comparing a new metaheuristic to existing ones, it is advantageous to test on the problem instances already tested by previous papers. Then, results will be comparable on a by-instance basis, allowing relative gap calculations between the two heuristics. A Testbed makes possible follow a formalized methodology and reproduce tests for further analysis and comparison. It seems natural that one of the most important parts of a comparison among heuristics is a testbed\cite{GendreauMichelandPotvin2010}.

This paper addresses the problem of comparing the use of several metaheuristics in search based tests. In this paper, we propose a flexible testbed tool to evaluate various diversity combining metaheuristics in search based software testing. A tool named IAdapter (github.com/naubergois/newiadapter), a JMeter plugin for performing search-based load tests, was extended \cite{Gois2016}. The IAdapter Testbed is an open-source tool that provides  tools for search based test research. This tool emulates test scenarios in a controled environment using mock objects and implementing performance antipatterns. Differently from patterns, antipatterns look at the negative features of a software system and describe commonly occurring solutions to problems that generate negative consequences.

Two experiments were conducted to validate the proposed tool. The experiments uses genetic, algorithms, tabu search, simulated annealing and an hybrid approach proposed by Gois et al. \cite{Gois2016}.

The remainder of the paper is organized as follows. Section 2 presents a brief introduction about load, performance, and stress tests. Section 3 presents concepts about the workload model. Section 4 presents details features about common performance antipatterns. Section 5 presents concepts about search based tests. Section 6 presents concepts about metaheuristic algorithms. Section 7 presents concepts about IAdapter Testbed. Section 8 shows the results of two experiments performed using the IAdapter plugin.  Conclusions and further work are presented in Section 10.
