\documentclass[]{article}
\usepackage{booktabs}
\usepackage{tikz}

\usetikzlibrary{arrows,positioning,fit,shapes,calc,automata}
\author{Luke Smith}
% Used example code from http://www.texample.net/tikz/examples/ 
\begin{document}

\tikzset{
every circ/.style = {fill=white!100},
circ/.style = {circle, dashed, draw, minimum size=6mm},
every box/.style = {},
box/.style = {rectangle, draw=black!80, ultra thick,
    minimum size = 6mm, every box},
gbox/.style = {fill=gray!80, box},
wbox/.style = {fill=white!100, box},
bbox/.style = {fill=blue!50, box},
arrow/.style = {<-,>=stealth',shorten >=1pt,auto,node distance=3cm,
  thick},
system/.style = {rectangle, fill=white!100, ultra thick, draw=black!80,
            minimum height=60mm, minimum width=4cm,outer sep=0pt}}

\begin{figure}[h]
\centering
\pgfdeclarelayer{background}
\pgfdeclarelayer{foreground}
\pgfsetlayers{background,main,foreground}
\begin{tikzpicture}[node distance=1cm, on grid]
    \begin{pgfonlayer}{background}
        \node [system] (system) at (0,3){};
    \end{pgfonlayer}
        \node [circ] (magnify) [minimum size=22mm] at ($(system.south)!.5!(system.north)$) {};
	\draw [densely dashed] (-2,4) to (-1,2.6) node[auto,midway] {};
	\draw [densely dashed] (-2,4) to (0,4.1) node[auto,midway] {};
	\node [gbox] (Step 1) [minimum height=24mm] at (-4,3) {Step 1};
    \node [above] at (Step 1.north) {Schema ($S$)};
    \node [wbox] (Step 2) at( $(system.south west)!.5!(system.north west)$)
        {Step 2} edge [arrow] (Step 1.east);
    \node [wbox] (Step 3) at ( $(system.south)!.8!(system.north)$)
        {Step 3} edge [arrow] (Step 2);
    \node [above] at (Step 3.north) {};
    \node [wbox] (Step 5) at ( $(system.south east)!.5!(system.north east)$)
        {Step 5} edge  [arrow] (Step 3);
    \path (Step 5)++(2,0) node [gbox] (Step 6) [minimum height=48mm] 
        {Step 6} edge  [arrow] (Step 5);
    \node [above] at (Step 6.north) {Doubled Schema ($S'$)};
    \node [bbox] (Step 41) [minimum size=4mm] at
    ($(system.west)!.4!(system.east)$) {} edge  [arrow] (Step 2);
    \node [bbox] (Step 42) [minimum size=4mm] at
    ($(system.south)!.2!(system.north)!.5!(system.west)!.5!(system.east)$)
    {} edge  [arrow] (Step 41);
    \node [wbox] (Step 5) at ( $(system.south east)!.5!(system.north east)$)
        {Step 5} edge  [arrow] (Step 42);
    \node [bbox] (Step 43) [minimum size=4mm] at
    ($(system.south)!.8!(system.north)!.5!(system.west)!.5!(system.east)$)
    {} edge  [arrow] (Step 41);
    \node [wbox] (Step 5) at ( $(system.south east)!.5!(system.north east)$)
        {Step 5} edge  [arrow] (Step 43);
    \node [above] at (Step 41.north) {Step 4};
    \node [wbox] (Step 3') at ( $(system.south)!.2!(system.north)$)
        {Step 3} edge  [arrow] (Step 2);
    \node [wbox] (Step 5) at ( $(system.south east)!.5!(system.north east)$)
        {Step 5} edge  [arrow] (Step 3');
    \node [above] at (Step 3'.north) {};

\end{tikzpicture}
\\[2em]
\begin{tabular}{@{}p{14mm}@{}p{18mm}@{}p{8.8cm}}
    \toprule
    & \textbf {Step} & \textbf {Result} \\
        \midrule
        Step 1 & Input &  Original Schema to be doubled.\\
        Step 2 & Split &  Cut the Schema into sections based on attributes.\\
        Step 3 & Filter &  If it is not the current attribute to be
        doubled, ignore it.\\
        Step 4 & Double &  Take the attribute and generate a semantic double \\
        Step 5 & Merge & Stitch the pieces back together (including new
        attribute).\\
        Step 6 & Output & Return the doubled schema.\\
        \bottomrule
    \end{tabular}\label{tab:trt_datd}
    \caption{The framework for doubling experiments with relational
    schemas}
\end{figure}

\end{document}

